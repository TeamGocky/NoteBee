\documentclass{sig-alt-release2}
\usepackage{url}
\usepackage{color}
\usepackage{graphics,graphicx}
\usepackage{epsfig}
\usepackage{epstopdf}
\usepackage{colortbl}
\usepackage{multirow}
\usepackage{booktabs}
\usepackage{ifthen} 
\usepackage[parfill]{parskip}
\usepackage[table]{xcolor}
\usepackage{array}
\usepackage{float}
\newcolumntype{L}{>{\centering\arraybackslash}m{0.18\textwidth}}
\newcommand{\highPrio}{\cellcolor{green}High}
\newcommand{\medPrio}{\cellcolor{orange}Medium}
\newcommand{\lowPrio}{\cellcolor{yellow}Low}
\newcommand{\vLowPrio}{\cellcolor{yellow!50}Very Low}
\newcommand{\todo}[1]{\textcolor{red}{#1}}
\def\newblock{\hskip .11em plus .33em minus .07em}
\conferenceinfo{DIM3} {2013, Glasgow, UK} 
\CopyrightYear{2010}
\clubpenalty=10000
\widowpenalty = 10000
\title{CodeBuzz}
\author{
\alignauthor
    Craig McLaughlin\\
    Gordon Reid\\
    Michael Dyson\\
	\affaddr{Team Gocky}\\
	\affaddr{DIM3}\\
	\affaddr{Student no.s}\\
    \email{\{1002524M,1002536R,1007389D\}@students.glasgow.ac.uk}
}
\begin{document}

\maketitle

\begin{abstract}

This report documents the specification and design of CodeBuzz,
a web-application for posting code snippets. The aim of the application is
to facilitate programming language learning by exposing beginners to
example programs written by industry experts and academics, and to
aid reuse by providing a store of solutions to commonly recurring
problems in software engineering. 
\end{abstract}

\section{Aim of Application}

CodeBuzz is a code snippet oriented application that allows users to submit code
in a variety of programming languages. This code is categorised based on the
code's language, and function (e.g. sorting algorithm). This categorisation
will allow other users to search for code snippets (e.g. Python Bubble Sort)
and have returned to them a list of matching snippets. These can be ordered
by popularity or user rating.

When a user views a code snippet they have the option to copy the snippet into
their clipboard or comment and rate the snippet.

The application can be used by novice coders looking for examples of common
language functions and can be used by intermediate/expert programmers to supply
their own examples, and rate others. Academics may also find this of use as
it can provide a convenient source of example code (which may or may not be
good code, both useful in this context).

\subsection{Functionality}
\label{sec:functionality}

The site will operate similar to any collaborative web-based application
where users of the site need not be registered in order to contribute
to the site content. The functionality is detailed in the lists below.

The following functionality is available to both registered and
non-registered users:

\begin{itemize}
\item Code snippets have their syntax highlighted.
\item Snippet hit rate/popularity.
\item Categorised code snippets by language and function.
\item Ability to search for code snippets based on language and
function.
\item Textual copy-to-clipboard.
\item Downloadable source code.
\item Links to other code snippets.
\end{itemize}

The following functionality is only available to registered users:
\label{sec:restrict}

\begin{itemize}
\item User can comment on the code snippets, e.g. reviewing the
correctness/usefulness of the code snippet.
\item User can rate a code snippet on a five star scale.
\item User profile which stores bookmarked snippets.
\item Social network integration (such as Twitter).
\end{itemize}

The design goals and functionality requirements are definitely achievable
within the time frame for the project, even with our lack of experience in
Django and web development. Even though the goals are realistically achievable
there was little sacrifice of complexity as the application has a broad use of
web technologies and includes an external API, Twitter.

\section{Client Interface}

\subsection{Wireframes}

Figure \ref{fig:blankPage} shows what the user will be greeted with when they
first access the page. The large blank text box is there to try and have the
user contribute to the code database.

Figure \ref{fig:anonsnippet} shows an anonymous snippet being posted by a
visitor to the website who has chosen not to log in. An example of the basic
layout and the syntax highlighting capabilities are shown in the figure.

Figure \ref{fig:joe} shows a logged in user posting a snippet. The user's
name is shown and there are also links to his profile, bookmarks, and other
user-specific data.

Figure \ref{fig:LoggedInViewJane} shows a user called Jane logged in and viewing
a snippet posted by an anonymous user. There is rating and comments section
with a text box for Jane to add in her own comment if she wishes. The page
scrolls as the comments are not necessarily immediately available above the
fold.

Figure \ref{fig:viewCategories} shows a user called Jane logged in and viewing
the snippet categories. Snippets fit into two categories: language, and type.
The view is identical if the user isn't logged in, with the obvious exception
of the upper bar. The upper bar would look like what was shown in Figure
\ref{fig:anonsnippet}.

Figure \ref{fig:viewAfterSubmission} shows a user called Joe viewing a snippet
he has just posted.

\subsection{Design Goals}

The number one design goal is to make the user interface minimalist
such that the user is not overwhelmed by the application. As can be
seen in the wire frames described above, the application will have a simple
home screen both for registered and non-registered users.

\subsection{A Walk through: Submit a code snippet in C}
\label{sec:walkthrough}

A simple walk through for the user Joe (see Section~\ref{sec:joe}
entering his C code snippet as shown in Figure~\ref{fig:joe} is given
below:

\begin{enumerate}
\item Joe is presented with the blank home screen and is not currently logged
in.
\item Joe logs in using his user details.
\item Joe is presented with the home screen again.
\item Joe selects the C programming language from the language drop-down
menu.
\item Joe proceeds to write his C code into the main window.
\item Joe selects the `Data Structures' category from the category
drop-down menu.
\item Joe is finished creating his code snippet and clicks the `Submit'
button.
\item Joe is presented with his submitted snippet.
\item Joe logs out.
\end{enumerate}

Note that the walk through above is applicable to a non-registered user
as well. They can miss out the login/logout steps. For restrictions on
non-registered users see Section~\ref{sec:restrict}.

This walk through has highlighted a number of interactions between the
user and the application. It is important to note the order in which
the steps are carried out, particularly the selection of the
programming language and the entering of the code. The order is
important because the selection will dynamically set the syntax
highlighting mode in the application to the language specified.
The text that the user then enters is highlighted according to the
language selected. This highlighting is performed on-demand. % (!)
However, the application will not enforce this ordering on the user.

\section{Personas}

\subsection{Barry: The Noob Programmer}
\label{sec:barry} 

Barry is sixteen, studying a computing based course and is looking to
learn a particular language. Barry's programming ability is, at best,
amateur. He relies heavily on introductory texts and Internet forums
since his school does not provide learning materials for the language
of interest.

\subsubsection{Goals}

\begin{itemize}
\item Finding code snippets to perform a specific programming task,
e.g. reading from a file.
\item Looking for code on a particular language.
\item To download/copy the code snippet for integration into their
program.
\item Comment on a snippet to ask questions if understanding is low.
\end{itemize}

\subsubsection{Behaviours}

\begin{itemize}
\item Curiosity towards others' code for inspiration for their own code
development. Looking for ideas on, `How it's done', etcetera.
\item Impatient regarding the `slowness' of their learning, wanting to
get their application up and running as soon as possible. Quick and
dirty.
\end{itemize}

\subsubsection{Likes}

\begin{itemize}
\item Syntax-highlighted environments that help his inexperienced brain
comprehend what is going on.
\item When code works out of the box.
\item Easy-to-read code, using simple constructs and ideas so that it
is simple to digest.
\end{itemize}

\subsubsection{Dislikes}

\begin{itemize}
\item Being unable to find the desired code snippet, or one which is
too complex/advanced for their current abilities.
\item Being overwhelmed by large number of advanced search capabilities.
\end{itemize}

\subsection{Joe: The Academic}
\label{sec:joe}

Joe is a University Professor teaching software engineering at
one of the top universities in his country. He wishes to contribute
code to a publicly accessible forum for his research area to entice
people to join his research project entitled
\textit{Software Elasticity in Safety-Critical Systems}. He also wishes
to publish code for courses he teaches to promote a social coding
environment for use by his students. His aim is to get all his students
using CodeBuzz to submit coursework, share interesting code samples,
etc, instead of other social networking sites so they stop
wasting their lives having meaningless conversations about cats on
invisible bikes with strangers.

\subsubsection{Goals}

\begin{itemize}
\item Add source code simulation of software degradation and how it's
entropy decreases over time in a safety-critical environment.
\item Look for crowd-sourced examples of how, or how not to, code a certain
programming function for use in class.
\item Use as a platform for student-submitted code that can be peer-reviewed.
\item Looks to aid the learning of students and others in their programming.
\end{itemize}

\subsubsection{Behaviours}

\begin{itemize}
\item Comment/rate student-submitted code as part of the learning process.
\item Upload examples of standard/optimised solutions to coding problems.
\item Submits high quality code for re-use and to contribute to the Open
Source community.
\item Gives low ratings and disapproving comments to incorrect,
inefficient or ugly code snippets to discourage beginners from
adopting poor techniques or bad programming habits.
\end{itemize}

\subsubsection{Likes}

\begin{itemize}
\item The ability to review code snippets via comments and ratings.
\item Being able to provide sample high-quality, working solutions.
\end{itemize}

\subsubsection{Dislikes}

\begin{itemize}
\item Poor quality code being passed off as working.
\item Students not keeping up with work given.
\end{itemize}

\subsection{Jane: Experienced Programmer}

Jane works in industry on medium-large commercial software projects. She has 
working knowledge of multiple programming languages and is well versed in 
different paradigms and making use of good software engineering concepts such as 
design patterns.

\subsubsection{Goals}

\begin{itemize}
\item Scout out potential future job candidates.
\item Suggest optimisations and changes to a user's submitted code.
\end{itemize}

\subsubsection{Behaviours}

\begin{itemize}
\item Becomes frustrated with other's mistakes however wishes to aid the
learning of individuals so they can become more proficient in programming.
\end{itemize}

\subsubsection{Likes}

\begin{itemize}
\item Seeing potential in other people for future job roles.
\end{itemize}

\subsubsection{Dislikes}

\begin{itemize}
\item Poor quality code being passed off as working.
\item Repeatedly being asked simple questions or seeing the same fundamental
errors.
\item Beginners that do not `R.T.F.M.'.
\end{itemize}

\begin{table*}
\begin{tabular}{*{6}{|L}|}
\hline
I want to... & Overall Priority & Barry & Bill & Jane\\
\hline
Search solution code & \highPrio & \highPrio & \medPrio & \lowPrio \\
\hline
Download code snippet & \highPrio & \highPrio & \medPrio & \lowPrio \\
\hline
Contribute quality code & \highPrio & \lowPrio & \highPrio & \highPrio \\
\hline
Rate a code snippet & \highPrio & \lowPrio & \highPrio & \medPrio \\
\hline
Comment on a snippet & \highPrio & \medPrio & \highPrio & \lowPrio \\
\hline
Bookmark a snippet & \medPrio & \medPrio & \highPrio & \medPrio \\
\hline
Submit code for review & \medPrio & \highPrio & \vLowPrio & \vLowPrio \\
\hline
Share code externally & \lowPrio & \medPrio & \vLowPrio & \vLowPrio \\
\hline
View highly rated code & \medPrio & \highPrio & \medPrio & \lowPrio \\
\hline
View code hit count & \lowPrio & \lowPrio & \vLowPrio & \vLowPrio \\
\hline
\end{tabular}
\caption{User Needs Matrix}
\end{table*}

\section{Application Architecture}

\begin{itemize}
\item	N-Tier Architecture Diagram

The architecture for the application is a 3-tier architecture as shown
in Figure~\ref{fig:ntier}.

\item	i.e. data flow diagram between the interface/client, middle ware,
and backend services/data repos


\item	Describe the data model i.e. what data needs to be stored or
persisted by the application?



\item	What are the relationships within the data model.
\item	i.e. use ER diagram and explain.

The ER diagram for the application is shown in Figure~\ref{fig:erdiag}.



\item	Describe the backend services used (if any).

Twitter integration...

\item	Reflective Questions: 
\item	How have you ensured that there is a separation of concerns?

Django provides a framework that forces the developer to think in terms
of Model-View-Template. However, this constraint is not restrictive but
enforces a healthy separation of concerns between the components of
the application e.g. URLs, presentation.

\item	What other technology could have been used instead of django?

Another web framework.

\item	What are the advantages of using a Web Application Framework
over other technology?

The framework provides a lot of services common to all web applications
``out of the box'', meaning that the developer can focus on features
unique to their application rather than spending time implementing their
own login and authentication system, etc.

\item	And, what are the disadvantages?

\end{itemize}

\section{Message Parsing}

Figures~\ref{fig:loginseqdiag} to~\ref{fig:postseqdiag} illustrate a
sequences for login, posting a snippet and logout for the
walkthrough in Section~\ref{sec:walkthrough}. The responses from the
middleware are HTML5 responses (some have client-side Ajax/jQuery/JSON
scripts included in them). All page requests require a ORM request to
the Database to retrieve the top five most recent and highest rated
code snippets.

\begin{itemize}
\item Describe the messages that are parsed back and forth through
the application.
\item	For the main transactions - describe the payload of the messages 
\item	i.e. What are the contents of the messages? i.e. include
sample XML, XHTML, JSON, etc of one or two messages.
\item	What is the format of the messages? 
\item	Why this format? 
\item	What other formats could be used, what are the advantages
and disadvantages of these other formats?
\end{itemize}

\section{Implementation Notes}

\subsection{Views}

We implemented the obligatory register/login/logout views and a view called
`user\_view' that displays some public information in a profile page such as
the user's most recent and top rated snippets, languages, and a summary of
activity. This is available for viewing by either clicking on a user's name
or going to your own `Your Profile' link when logged in.

We also implemented views listed below:

\begin{itemize}
\item Homepage - The homepage primarily acts as the page where content
can be added. The user is greeted with a message asking them to post a snippet
which a text box already in focus, waiting for input.

\item View Snippet - This is the complementary view to the 
homepage. This view displays the title, a brief summary, and the code for a 
code snippet. This page also gives a logged in user the ability to rate,
comment, or tweet the snippet.

\item Search  -These two views are our
two search mechanisms. The first allows a standard keyword search on snippet
names. The latter allows for keyword, category, and language specific searching.

\item Browse -
These snippets allow the user to search all snippets that fall under a certain
language or category.
\end{itemize}

\subsection{URL Mapping}

Our application is comprised of three apps: accounts, registration, and
codesnippet. 

Registration has a single URL `/registration/' that handles the
simple form required to become a member of the website.

Accounts handles the login, logout, and profile view via the URLs:
`/accounts/login', `/accounts/logout', and `/accounts/view/<user id>'
respectively.

The Codesnippet app's URL mappings are equally as readable. All URLs, again,
begin with `/codesnippet/' (not required to get to homepage). Postfixes for
search and advanced search are `search/' and `advanced\_search/' respectively.
This is similar for browse (by language/category) with
 `/browse/language|category/<language>|<category>/'. Also, to view a snippet
the URL the postfix is `view/<snippetID>/'.

\subsection{External Service}

Our sole external service is with the Twitter API. A Python package
python-twitter is available that prevents the developers wishing to use the
API from having to write any handler code.

\subsection{Functionality implemented and known issues}

All primary functionality stated earlier in the report has been implemented and
there are no known issues at time of writing.

\subsection{Technologies Used}

\begin{itemize}
\item HTML (HTML5 backwards compatible with HTML4)
\item CSS (Mostly CSS2 with some non-essential CSS3 such as border-radius)
\item Javascript/jQuery - Required to work CodeMirror's syntax highlighting,
for detecting a low resolution display and changing the CSS, and for the JSON
rating system.
\item Python + Django
\item SQLite (although should be replaced by MySQL or similar)
\end{itemize}

\section{Reflective Summary}

	What have you learnt through the process of development? 
	How did the application of frameworks help or hinder your progress? 
	What problems did you encounter? 
	What were your major achievements?


\section{Summary and Future Work}

CodeBuzz is a code snippet oriented application that allows users to submit code
in a variety of programming languages. Through categories and languages,
the site provides users with the ability to search for code snippets.
Code snippets can be rated and, if a user has registered, commented on.
Most of the functionality specified in the
Section~\ref{sec:functionality} has been implemented and the appearance
of the application adheres very closely to the wireframes.

Allowing languages or categories to be added by users of the site is
currently not supported.
Currently, the application only supports Twitter but future work could
extend on this to support Facebook integration. A language syntax
detection mechanism could developed.

\begin{itemize}
\item Include a list or table of all the technologies, standards,
and protocols that will be required.
\end{itemize}

\section{Acknowledgements}

Thanks to Dr. Leif Azzopardi for his input and suggestions.

Thanks to Euan Freeman for being a repository of knowledge and for
suggesting the unique rating system for the application.

Thanks to Professor Joseph Sventek for letting us use his image for the
rating system.

Thanks to Marijn Haverbeke for developing and maintaining the excellent
CodeMirror~\footnote{http://codemirror.net/} component that was used to
provide the syntax highlighting of code snippets.

Thanks to the jQuery~\footnote{http://jquery.com/} team for making
JavaScript usable.

Last but certainly not least, thanks must go to Django for providing
a fantastic web application framework that was a joy to work with.
\section{References}

\begin{figure*}
\includegraphics[width=\textwidth]{../imgs/InitialHomepageHorz.png}
\caption{Blank Homepage}
\label{fig:blankPage}
\end{figure*}

\begin{figure*}
\includegraphics[width=\textwidth]{../imgs/homepageWireFrameGRHorz.png}
\caption{Anonymous snippet being posted.}
\label{fig:anonsnippet}
\end{figure*}

\begin{figure*}
\includegraphics[width=\textwidth]{../imgs/CCodeSnippetHorz.png}
\caption{Joe's home screen with his code snippet}
\label{fig:joe}
\end{figure*}

\begin{figure*}
\includegraphics[width=\textwidth]{../imgs/LoggedInViewSnippetHorz.png}
\caption{Jane viewing a posted snippet.}
\label{fig:LoggedInViewJane}
\end{figure*}

\begin{figure*}
\includegraphics[width=\textwidth]{../imgs/viewingCategoriesHorz.png}
\caption{Jane looking at the categories of snippet.}
\label{fig:viewCategories}
\end{figure*}

\begin{figure*}
\includegraphics[width=\textwidth]{../imgs/CSnippetAfterSubmissionHorz.png}
\caption{Joe looking at his most recent snippet after submission.}
\label{fig:viewAfterSubmission}
\end{figure*}

\begin{figure*}
\centering
\includegraphics[scale=0.6]{../imgs/N-TierArchitecture.png}
\caption{3-Tier Architecture for application}
\label{fig:ntier}
\end{figure*}

\begin{figure*}
\centering
\includegraphics[scale=0.6]{../imgs/ERDiagram.png}
\caption{Entity-Relationship Diagram for application}
\label{fig:erdiag}
\end{figure*}

\begin{figure*}
\centering
\includegraphics[scale=0.6]{../imgs/walkthroughseqdiag-login.png}
\caption{Login Sequence diagram for Joe's Walkthrough described in
Section~\ref{sec:walkthrough} steps 1-3}
\label{fig:loginseqdiag}
\end{figure*}

\begin{figure*}
\centering
\includegraphics[scale=0.6]{../imgs/walkthroughseqdiag-postsnippet.png}
\caption{Post Snippet Sequence diagram for Joe's Walkthrough described in
Section~\ref{sec:walkthrough} steps 4-8}
\label{fig:postseqdiag}
\end{figure*}

\begin{figure*}
\centering
\includegraphics[scale=0.6]{../imgs/walkthroughseqdiag-logout.png}
\caption{Logout Sequence diagram for Joe's Walkthrough described in
Section~\ref{sec:walkthrough} step 9}
\label{fig:logoutseqdiag}
\end{figure*}

\end{document}  
