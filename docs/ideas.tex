\documentclass[11pt,a4paper]{article}
\usepackage[parfill]{parskip}
\usepackage{fullpage}
\usepackage{array}
\newcolumntype{L}{>{\centering\arraybackslash}m{3cm}}
\title{NoteBee}
\author{Team Gocky}
\begin{document}

\maketitle

\tableofcontents
\pagebreak

\section{Introduction}

NoteBee is a code snippet oriented application that allows users to submit code
in a variety of programming languages. This code is categorised based on the
code's language, and function (e.g. sorting algorithm). This categorisation
will allow other users to search for code snippets (e.g. Python Bubble Sort)
and have returned to them a list of matching snippets. These can be ordered
by popularity or user rating.

When a user views a code snippet they have the option to copy the snippet into
their clipboard or comment and rate the snippet.

The application can be used by novice coders looking for examples of common
language functions and can be used by intermediate/expert programmers to supply
their own examples, and rate others. Academics may also find this of use as
it can provide a convenient source of example code (which may or may not be
good code, both useful in this context).

The site will operate similar to any collaborative web-based application
where users of the site need not be registered in order to contribute
to the site content. However, only registered users will be allowed to
rate a snippet. Other benefits of a registered user account include
having links to all your submitted code snippets, and
favourite/bookmarked code snippets.

\section{Functionality}

\begin{itemize}
\item Code snippets have their syntax highlighted.
\item User comments on the correctness/usefulness of the code snippets.
\item User rating (in extension to the comments)
\item Note hit rate/popularity.
\item Categorised code snippets by language and function.
\item Ability to search for code snippets based on language and function.
\item Textual copy-to-clipboard.
\item Downloadable source code.
\item Social network integration.
\item Links to other code snippets.
\end{itemize}

\section{Personas}


\subsection{The Noob Programmer}
% CMCL went with some suggestions from
% http://www.steptwo.com.au/papers/kmc_personas/index.html
% eliminating the personal details and just detailing the following:
% user’s goals, behaviours, likes and dislikes. 

This person is a beginning or amateur programmer perhaps studying a
computing course although could be a hobbyist. Their behaviour includes:

\subsubsection{Goals}

\begin{itemize}
\item Finding code snippets to perform a specific programming task,
e.g. reading from a file.
\item Looking for code on a particular language.
\item To download/copy the code snippet for integration into their
program.
\end{itemize}

\subsubsection{Behaviours}

\begin{itemize}
\item Curiosity towards others code for inspiration for their own code
development. Looking for ideas on, "How it's done", etc.
\item Impatient regarding the "slowness" of their learning, wanting to
get their application up and running as soon as possible. Quick and
dirty.
\end{itemize}

\subsubsection{Likes}

\begin{itemize}
\item When code works out of the box.
\item Easy-to-read code, using simple constructs and ideas so that it
is simple to digest.
\end{itemize}

\subsubsection{Dislikes}

\begin{itemize}
\item Being unable to find the desired code snippet, or one which is
too complex/advanced for their current abilities.
\item Being overwhelmed by large number advanced search capabilities.
\end{itemize}

\subsection{The Academic}

This person is a University Professor, College Professor, or possibly high
school teacher teaching a computing-based course.

\subsubsection{Goals}

\begin{itemize}
\item Looking for crowd-sourced examples of how, or how not to, code a certain
programming function for use in class.
\item Use as a platform for student-submitted code that can be peer-reviewed.
\item Comment/rate student-submitted code as part of the learning process.
\item Upload examples of standard/optimised solutions to coding problems.
\end{itemize}

\subsubsection{Behaviours}
\begin{itemize}
\item Looking to aid the learning of students and others in their programming.
\item Interested in submitting high quality code for re-use and to contribute
to the Open Source community.
\end{itemize}

\subsubsection{Likes}
\begin{itemize}
\item The ability to review code snippets.
\item Being able to provide sample high-quality, working solutions.
\end{itemize}

\subsubsection{Dislikes}
\begin{itemize}
\item Poor quality code being passed off as working.
\end{itemize}

\subsection{Experienced Programmer}
\subsubsection{Goals}
\subsubsection{Behaviours}
\subsubsection{Likes}
\subsubsection{Dislikes}

\section{User Needs Matrix}

\begin{table}[h]
\begin{tabular}{*{5}{|L}|}
\hline
I want to... & Overall Priority & Noob Programmer 
& Academic & Experienced Programmer\\
\hline
Search solution code & 2 & 3 & 2 & 1 \\
\hline
Contribute quality code & 2 & 1 & 3 & 3 \\
\hline
\end{tabular}
\end{table}

\end{document}  
